\chapter{Moléculas Fotónicas}

La técnica de escritura de guías de onda descrita en el capítulo \ref{cap:fs} está restringida por la forma alargada y elíptica del tren de pulsos láser que se enfoca, lo que en consecuencia constriñe los acoplamientos interorbitales posibles \citep{interorbital}. Una posibilidad para añadir grados de libertad es fabricar dos guías de onda lo suficientemente cercanas entre sí de manera de hibridizar sus modos guiados, de manera análoga al principio físico que rige a las moléculas. Es por ello que en este capítulo se usará el concepto de moléculas fotonicas \citep{molecules}, y su aplicación para el estudio experimental de una red fotónica que presenta una doble transición de fase topológica \citep{SPSSH}.

\section{Autoestados del acoplador fotónico para distancias de separación arbitrarias}

Como se adelantó en la Sección~\ref{cap:CMT}, la teoría de modos acoplados (CMT) describe adecuadamente los sistemas fotónicos en estudio cuando la distancia entre guías de onda supera los 13\,$\mu$m. Para separaciones menores, el sistema debe considerarse como una única macroguía. La expansión en modos normales (EME), descrita en la Sección~\ref{cap:eme}, proporciona una herramienta numérica válida para ambos regímenes. Mediante simulaciones de pares de guías de onda a diferentes distancias, se caracterizó el comportamiento de sus autoestados. La Figura~\ref{fig:molecule-coup} muestra que el autoestado antisimétrico ($+-$) sólo existe para distancias mayores a 13\,$\mu$m, ya que por debajo de este umbral se convierte en un modo de radiaci\'on ($k_z^{+-} < k_0 n_0$). En contraste, el autoestado simétrico ($++$) persiste y puede describirse como una única entidad, lo que en esta tesis llamaremos \textit{molécula fotónica}.

\begin{figure}[H]
	\centering
	\includegraphics[width=0.50\linewidth]{codigo/dimol2/eigenvalues_vs_angle_1.png}
	\includegraphics[width=0.45\linewidth]{codigo/dimol3/eigenvalues_vs_wavelength.png}
	\caption[Propagaci\'on y acoplamientos en mol\'eculas fot\'onicas]{
		\textbf{Izquierda:} Constantes de propagación y acoplamientos en función de la distancia para modos fundamentales, calculados mediante EME. 
		\textbf{Derecha:} Constantes de acoplamiento en función de la longitud de onda.
		\label{fig:molecule-coup}}
\end{figure}

\section{Moléculas Fotónicas en Red SP-SSH}

Para la implementación experimental (sección \ref{cap:fs}) de una red que presente acoplamiento SP \citep{interorbital, SPSSH}, se utilizaron los dipolos horizontales de la sección anterior, obtenidos mediante moléculas fotónicas. Un preciso sintonizado de las constantes de propagación de los modos $s$ y $p$ permitió considerar un grado de libertad análogo al del espín del electrón (\textit{pseudoespín}). El Hamiltoniano $H$ de esta red \citep{SPSSH,toporusos}  es el siguiente

\begin{align*}
	H &= \sum_n \left[\frac{\delta\beta}{2} \left( b^*_{n, 1} b_{n, 1} - a^*_{n, 1} a_{n, 1} + b^*_{n, 2} b_{n, 2} - a^*_{n, 2} a_{n, 2} \right) +k_{ss, 2}a^*_{n, 2} a_{n, 1} -k_{pp, 2}b^*_{n, 2} b_{n, 1} \right. 
	\\	
	&+ k_{ss, 1} \left( a_{n-1, 2}^*a_{n, 1} + a_{n+1, 2}^*a_{n, 2} \right) - k_{pp, 1} \left( b_{n-1, 2}^*b_{n, 1} + b_{n+1, 2}^*b_{n, 2} \right) + k_{sp, 2} \left( a_{n, 2}^* b_{n, 1} - b_{n, 2}^* a_{n, 1} \right)
	\\
	&+ \left. k_{sp, 1} \left( a_{n-1, 2}^* b_{n, 1} - b_{n-1, 2}^* a_{n, 1} + a_{n+1, 1}^*b_{n, 2} - b_{n+1, 1}^* a_{n, 2} \right) \right] + c.c.
\end{align*}

\begin{figure}[H]
\centering
	\includegraphics[width=0.7\linewidth]{media/ssh_sp_model}
	\caption{Esquema de la red SP-SSH y los acoplamientos considerados.}
\end{figure}