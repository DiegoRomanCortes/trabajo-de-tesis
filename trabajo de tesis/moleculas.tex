\chapter{Moléculas Fotónicas}

La técnica de escritura de guías de onda descrita en el capítulo \ref{cap:fs} está restringida por la forma alargada y elíptica del tren de pulsos laser que se enfoca, lo que en consecuencia constriñe los acoplamientos interorbitales posibles \citep{interorbital}. Una posibilidad para añadir grados de libertad es fabricar dos guías de onda lo sificientemente cercanas entre sí de manera de hibridizar sus modos guiados, de manera análoga al principio físico que rige a las moléculas. Es por ello que en este capítulo se usará el concepto de moléculas fotonicas \citep{molecules}, y su aplicación para el estudio experimental de una red fotónica que presenta una doble transición de fase topológica \citep{SPSSH}.

\section{Autoestados del dímero fotónico para distancias de separación arbitrarias}

