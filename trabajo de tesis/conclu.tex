\chapter{Conclusiones \label{cap:conclu}}

Esta tesis ha investigado sistemáticamente las propiedades de redes fotónicas discretas con acoplamientos distintos al de modos $S$, combinando desarrollos teóricos, simulaciones numéricas y validación experimental. Los resultados principales se organizan en torno a dos sistemas fundamentales estudiados a lo largo de esta investigación.

En el Capítulo~\ref{cap:invisibility} se caracterizó una red fotónica tipo panal de abeja compuesta por modos $p_y$, donde se demostró que el ángulo de invisibilidad induce un régimen de bandas cuasi-planas alrededor de $\theta \approx 0.6$ radianes. Mediante un análisis exhaustivo que incluyó correcciones por no-ortogonalidad y acoplamientos de largo alcance, se estableció que el confinamiento tipo Aharonov-Bohm persiste incluso en presencia de estas perturbaciones. Particularmente, se observó que los efectos no-ortogonales son necesarios para describir correctamente la dinámica de localización en los bordes de la red, un fenómeno cuantificado mediante el grado de participación inversa (IPR) y confirmado experimentalmente.

El Capítulo~\ref{cap:molecules} abordó el estudio de redes SP-SSH con moléculas fotónicas, revelando que la dimerización fotónica permite implementar dos transiciones topológicas consecutivas controladas independientemente. La hibridización de modos $S$ y $P$ en estas estructuras genera estados moleculares con polarización de bulto cuantizada, cuya existencia fue verificada mediante un protocolo experimental que correlaciona la razón $I_{\text{borde}}/I_{\text{total}}$ con las predicciones numéricas. Este sistema demostró además la posibilidad de crear estados localizados de borde, similares a los observados en cadenas SSH unidimensionales pero con el ingrediente adicional de la hibridación $SP$.

Las contribuciones metodológicas de esta tesis incluyen el desarrollo de un enfoque numérico continuo que supera las limitaciones fundamentales de la teoría de modos acoplados para sistemas con índices de refracción arbitrarios. Éstas se complementaron con técnicas experimentales para la caracterización de redes fotónicas multiorbitales, particularmente en lo que respecta a la medición de parámetros de acoplamiento y la detección de transiciones topológicas.

Los hallazgos presentados abren el camino para diversas aplicaciones, particularmente en el diseño de dispositivos fotónicos que aprovechen los mecanismos de localización inducidos por geometría y transiciones topológicas. Los resultados sugieren además la posibilidad de extender este marco teórico-experimental al estudio de sistemas con mayor complejidad orbital como sistemas con momentum angular orbital \citep{OAMCaging}.