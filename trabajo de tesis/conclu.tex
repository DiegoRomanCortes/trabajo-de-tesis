\chapter{Conclusiones}
\label{cap:conclu}

Esta tesis ha investigado sistemáticamente las propiedades de redes fotónicas discretas con modos transversales no convencionales, centrándose en configuraciones que incluyen modos \( P \), acoplamientos interorbitales \( S\hspace*{-0.3ex}P \), y acoplamientos evanescentes no simétricos. La investigación combinó herramientas analíticas, simulaciones numéricas y validación experimental para abordar diferentes manifestaciones de la estructura orbital de la luz guiada.

En el Capítulo~\ref{cap:invisibility}, se caracterizó una red fotónica tipo panal de abeja compuesta por modos \( p_y \), demostrando que el \textit{ángulo de invisibilidad} induce un régimen de bandas cuasi-planas alrededor de \( \theta \approx 0.56 \) radianes. Mediante un análisis detallado que incorporó correcciones por no ortogonalidad y acoplamientos de largo alcance, se estableció que el confinamiento tipo Aharonov-Bohm persiste aún en presencia de estas perturbaciones. En particular, se comprobó que los efectos de no ortogonalidad son esenciales para describir correctamente la localización en los bordes de la red, lo cual fue cuantificado mediante el inverso del grado de participación (IPR) y corroborado experimentalmente.

El Capítulo~\ref{cap:asymmetric} abordó el fenómeno de acoplamiento evanescente no simétrico en dímeros fotónicos con diferentes contrastes de índice de refracción. Se presentó una generalización de la teoría de modos acoplados que incluye términos de no ortogonalidad y permite modelar correctamente la dinámica direccional observada. Experimentalmente, se verificó un desbalance en la transferencia de energía dependiente del canal de entrada, en acuerdo con las predicciones teóricas basadas en matrices de acoplamiento efectivas no simétricas.

En el Capítulo~\ref{cap:molecules}, se estudió la implementación de redes SP-SSH construidas a partir de moléculas fotónicas. Se demostró que la dimerización estructural, combinada con la hibridación entre modos \( S \) y \( P \), da lugar a una doble transición de fase topológica controlable de forma independiente. Se identificaron fases con distinta polarización de bulto cuantizada, y se diseñó un protocolo experimental para detectar modos de borde mediante la razón \( I_{\text{borde}} / I_{\text{total}} \), en excelente acuerdo con simulaciones numéricas. Este modelo extiende la física del sistema SSH unidimensional al contexto multiorbital, revelando nuevos mecanismos de localización topológica mediados por grados de libertad internos.

Desde el punto de vista metodológico, esta tesis desarrolló un método numérico continuo, basado en la expansión en modos normales (EME), que permite extender la teoría de modos acoplados en el régimen de guías próximas o contrastes de índice de refracción arbitrarios. Este enfoque fue complementado con técnicas experimentales de fabricación y caracterización de redes multiorbitales que permitieron la calibración de constantes de propagación de modos guiados distintos.

Los resultados obtenidos abren nuevas posibilidades en el diseño de dispositivos fotónicos que explotan mecanismos de localización y control basados en geometría, simetría transversal y topología. Asimismo, se establece un marco teórico-experimental adaptable a plataformas más complejas, por ejemplo, en sistemas con momentum angular orbital \citep{OAMCaging}.

