\chapter{Métodos numéricos}
A partir de las ecuaciones (\ref{eqn:helmholz}) y (\ref{eqn:helmholzH}), se puede aplicar la aproximación de guiaje débil para despreciar el lado derecho de ambas ecuaciones a fin de ignorar el efecto cruzado entre componentes de los campos. Las ecuaciones resultantes son del tipo Helmholtz:
\begin{equation}
	\left[\nabla^2 + k_0^2 n^2(\textbf{r})\right]\Psi(\textbf{r}) = 0. \label{eqn:helmholtznum}
\end{equation}
La ecuación (\ref{eqn:helmholtznum}) es la base de todos los métodos numéricos utilizados en esta tesis.
\section{Expansión en modos normales}
Este método numérico es úitl cuando los sistemas fotónicos en estudio son invariantes en la dirección de propagación $z$. Esto es, $n(\textbf{r})\equiv n(x,y)$. La soluciones de la ecuación (\ref{eqn:helmholtznum}) se pueden expandir en ondas planas con perfiles transversales: $\Psi(\textbf{r}) = \Psi(x,y) \exp({i\beta z})$. Con ésto, cada modo transversal $\nu$ debe cumplir la siguiente ecuación a resolver numéricamente:
\begin{equation}
	\left[\nabla_\perp^2 + k_0^2 n^2(x,y)\right]\Psi_\nu(x,y) = \beta_\nu^2\Psi_\nu(x,y) , \quad\text{ con } \nabla_\perp^2 \equiv \frac{\partial^2}{\partial x^2} + \frac{\partial^2}{\partial y^2}.
	\label{eqn:eme}
\end{equation}
Y el campo total propagado es una combinación lineal de los modos $\Psi_\nu$: 
\begin{equation}
	\Psi(\textbf{r}) = \sum_\nu a_\nu \Psi_\nu(x,y) e^{i\beta_\nu z}, \quad\text{ con } a_\nu \propto \Psi_\nu(x,y) \cdot \Psi(x, y, z=0).
\end{equation}
En vez de integrar directamente la ecuación de valores propios (\ref{eqn:eme}), la estrategia será discretizar el espacio y aproximar al operador Laplaciano transversal $\nabla_\perp^2$ como una matriz, pues $$\frac{\partial^2 \Psi(x,y)}{\partial x^2} \sim \frac{\Psi[i+1,j]-2\Psi[i,j]+\Psi[i-1,j]}{\Delta x ^2}.
$$
Esto es, la matriz será una suma de Kronecker de dos matrices tridiagonales con valores $-2$ en la diagonal y $-1$ fuera de ella, con un prefactor de $1/\Delta x^2$ o $1/\Delta y^2$.
Prácticamente toda la matriz es nula (\textit{sparse-like}), por lo que es posible optimizar el proceso de cómputo al utilizar la librería de Python \href{https://docs.scipy.org/doc/scipy/reference/sparse.linalg.html}{\color{magenta}\texttt{scipy.sparse.linalg}}, especialmente diseñada para el álgebra lineal de matrices de escasos elementos. El anexo \ref{sec:codigohelmholtz} contiene una implementación de este algoritmo en Python bajo \href{https://www.gnu.org/licenses/gpl-3.0.html}{\color{magenta}licencia GNU GPL v3}.
\section{Beam Propagation Method} 

Otra forma de abordar la resolución numérica de la ecuación (\ref{eqn:helmholtznum}) consiste en separar el campo en su envolvente lenta y una fase rápidamente oscilante (en el rango visible, $k_0 \sim 10 ^{7} m^{-1}$): $\Psi(\textbf{r}) = \phi(x,y,z)\exp(ik_0 n_0 z)$. Luego de reemplazar en la ecuación (\ref{eqn:helmholtznum}) se obtiene la ecuación óptica de Schrödinger \citep{paraxialschrodinger}.

\begin{equation}
	2ik_0 n_0\frac{\partial }{\partial z}\phi(x,y, z) = - \left[\nabla_\perp^2 + k_0^2 (n^2(\textbf{r})-n_0^2)\right]\phi(x,y, z), \label{eqn:bpmescalar}
\end{equation} 
donde se ha utilizado la aproximación paraxial $\left| \frac{\partial^2 \phi}{\partial z^2} \right| \ll 2 k_0 n_0\left| \frac{\partial \phi}{\partial z} \right|$. Los algoritmos que resuelven la ecuación (\ref{eqn:bpmescalar}) son conocidos como \textit{Beam Propagation Method} escalares, utilizados ampliamente en esta área de investigación \cite{bics, interorbital, OAMCaging, vortex, bpm}.



\section{Desde teoría de modos acoplados}


