\chapter{Métodos numéricos}

\section{Expansión en modos normales}

\section{Beam Propagation Method} 


método numérico escalar conocido como \textit{Beam Propagation Method}, utilizado ampliamente en esta área de investigación \cite{bics, interorbital, OAMCaging, vortex, bpm}. Éste consiste en resolver numéricamente la ecuación
\begin{equation}
	2in_0k_0\frac{\partial}{\partial z}\psi(x,y,z) = \nabla_\perp^2 \psi (x,y,z) + \left(n^2-n_0^2\right)k_0^2 \psi (x,y,z), \label{eqn:paraxial}
\end{equation}
con $\psi(x,y,z)$, $n_0$, $n(x,y)$ y $k_0$ la envolvente de la componente horizontal campo eléctrico, el índice de refracción del material, el índice de refracción inducido y el número de onda en el vacío, respectivamente. 



\section{Desde teoría de modos acoplados}


