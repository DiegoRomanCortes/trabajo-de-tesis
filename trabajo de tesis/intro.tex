\chapter{Introducción}

Entre los premios Nobel en Física de la última década se encuentran varios que están estrechamente ligados a la óptica \cite{nobel}: por la generación de pulsos de luz ultra cortos (femtosegundos \cite{femto1} y luego attosegundos \cite{atto1, atto2, atto3}), por experimentos con fotones entrelazados \cite{photons1, photons2, photons3}, por la ideación de pinzas ópticas \cite{opticaltweezers} y por la invención de luces LED \cite{led1, led2, led3}. El estudio del comportamiento de la luz en diversos contextos ha permitido el posterior desarrollo tecnológico con aplicaciones industriales, en medicina, en comunicaciones e incluso militares. Una aplicación cotidiana es la fibra óptica, que actúa como una guía de onda para la luz y actualmente es el principal medio de transmisión de Internet en el mundo \cite{fibra2, fibra}. 

Numerosos de estos avances en el control de las propiedades de transporte de la luz se han visto propiciados por la técnica de escritura de guías de onda por láser de femtosegundos, la cual ha sido utilizada para la fabricación de redes fotónicas de variada índole \cite{femto, bics, lieb1, lieb2, artificialFB, FBdynamics, strain, dendritas, splitters}. Su importancia radica no sólo en emular situaciones de la física del sólido, tales como oscilaciones de Bloch \cite{BlochOsci}, localización de Anderson \cite{Anderson}, estados de banda plana \cite{lieb1, lieb2, artificialFB, FBdynamics} o topología \cite{obstopo, obsfloquet, topo1dphoto,toporusos}, sino que también en el estudio de fenómenos ópticos incluyendo no-linealidad tipo Kerr y su uso en la formación de solitones \cite{discretesolitons}, la posible propagación de luz cuántica \cite{qed, squeezed, topoquantum}, o su compatibilidad con la transmisión de información en la industria de las telecomunicaciones \cite{telecom}.

El enfoque de este proyecto será el estudio de redes fotónicas multiorbitales, donde dos fenómenos clave emergen: los efectos dipolares y el acoplamiento interorbital SP. El ángulo dipolar de invisibilidad \cite{Pmodecoupling} introduce un nuevo grado de libertad para controlar la localización de luz, como se demostrará en el capítulo \ref{cap:invisibility}. Paralelamente, la técnica de acoplamiento interorbital, que consiste en sintonizar las constantes de propagación del modo fundamental (S) con el primer modo excitado (P) mediante calibración precisa de las potencias de escritura láser \cite{interorbital}, permitirá implementar experimentalmente una red SP-SSH \citep{SPSSH}. 

