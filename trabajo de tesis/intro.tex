\chapter{Introducción}
Entre los premios Nobel en Física de la última década se encuentran varios que están estrechamente ligados a la óptica \cite{nobel}: por la generación de pulsos de luz ultra cortos (femtosegundos \cite{femto1} y luego attosegundos \cite{atto1, atto2, atto3}), por experimentos con fotones entrelazados \cite{photons1, photons2, photons3}, por la ideación de pinzas ópticas \cite{opticaltweezers} y por la invención de luces LED \cite{led1, led2, led3}. El estudio del comportamiento de la luz en diversos contextos ha permitido el posterior desarrollo tecnológico con aplicaciones industriales, en medicina, en comunicaciones e incluso militares. Una aplicación cotidiana es la fibra óptica, que actúa como una guía de onda para la luz y actualmente es el principal medio de transmisión de Internet en el mundo \cite{fibra2, fibra}. 
	
	Numerosos de estos avances en el control de las propiedades de transporte de la luz se han visto propiciados por la técnica de escritura de guías de onda por láser de femtosegundos, la cual ha permitido la fabricación de redes fotónicas de variada índole \cite{femto, bics, lieb1, lieb2, artificialFB, FBdynamics, strain, dendritas, splitters}. Su importancia radica no sólo en emular situaciones de la física del sólido, tales como oscilaciones de Bloch \cite{BlochOsci}, localización de Anderson \cite{Anderson}, estados de banda plana \cite{lieb1, lieb2, artificialFB, FBdynamics} o topología \cite{obstopo, obsfloquet, topo1dphoto,toporusos}, sino que también en el estudio de fenómenos ópticos incluyendo no-linealidad tipo Kerr y su uso en la formación de solitones \cite{discretesolitons}, la posible propagación de luz cuántica \cite{qed, squeezed, topoquantum}, o su compatibilidad con la transmisión de información en la industria de las telecomunicaciones \cite{telecom}.
	
	El enfoque de este proyecto será el estudio de redes fotónicas multiorbitales. Por ello será crucial incorporar la técnica de acoplamiento interorbital, que consiste en sintonizar las constantes de propagación de el modo fundamental de una guía monomodal (S) con el primer modo guiado excitado de una guía dimodal (P) mediante la calibración adecuada de las potencias de escritura, que inducen diferencias en los contrastes generados por la técnica de escritura por láser femtosegundos \cite{interorbital}.
	
	El llamado acoplamiento SP ha permitido el estudio de redes que presentan flujo magnético efectivo $\Phi = \pi$, el cual permite el transporte controlado de la luz \cite{OAMCaging, ABCaging}. Una aplicación directa de este fenómeno es la generación de guías de onda que admitan modos guiados de luz con momentum angular orbital (OAM) y la codificación de su carga topológica $\ell$ como medio para transmitir información \cite{oamapp, oamfree}. Se ha reportado a la fecha sólo la propagación de OAM mediante de redes fotónicas que presevan simetría $C_3$ \cite{OAMWG, vortex}. Sin embargo, el acoplamiento entre modos OAM en una red fotónica permitiría la generación de flujos magnéticos distintos de $0$ o $\pi$, lo que se reflejaría en una direccionalidad dependiente de la circulación propagante \cite{vortextrim, topoOAM}. Para ello será necesario introducir el concepto de ``moléculas fotónicas'' \cite{molecules} y estudiar su aplicación en redes fotónicas \cite{SPSSH}.