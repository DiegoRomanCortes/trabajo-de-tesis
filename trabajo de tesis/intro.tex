\chapter{Introducción}

Durante la última década, varios Premios Nobel en física han estado estrechamente ligados a la óptica \cite{nobel}: por la generación de pulsos ultracortos de luz —femtosegundos \cite{femto1} y luego attosegundos \cite{atto1, atto2, atto3}—, por experimentos pioneros con fotones entrelazados \cite{photons1, photons2, photons3}, por el desarrollo de pinzas ópticas \cite{opticaltweezers} y por la invención de LEDs de alta eficiencia \cite{led1, led2, led3}. Estos avances fundamentales han impulsado aplicaciones industriales, biomédicas, en telecomunicaciones e incluso en defensa. Una aplicación cotidiana destacada es la fibra óptica, que actúa como guía de onda para la luz y constituye hoy el principal medio de transmisión de datos a nivel global \cite{fibra2, fibra}.

Muchos de estos desarrollos se han visto potenciados por la técnica de \textit{escritura por láser de femtosegundos}, que permite fabricar redes fotónicas tridimensionales de guías de onda con precisión micrométrica. Esta tecnología ha habilitado la realización experimental de redes con geometrías complejas y grados de libertad artificiales, y ha sido utilizada para estudiar fenómenos como oscilaciones de Bloch \cite{BlochOsci}, localización de Anderson \cite{Anderson}, dinámica de bandas planas \cite{lieb1, lieb2, artificialFB, FBdynamics}, fases topológicas \cite{obstopo, obsfloquet, topo1dphoto, toporusos}, formación de solitones no lineales \cite{discretesolitons} e incluso la propagación de luz cuántica \cite{qed, squeezed, topoquantum}.

Dentro de este marco, el presente trabajo se enfoca en el estudio de redes fotónicas \textit{multiorbitales}, en las que la estructura interna de los modos transversales de la luz puede utilizarse como un nuevo recurso de control. Particularmente, se estudian los modos fundamentales de guía (denominados aquí $S$) y sus primeras excitaciones transversales antisimétricas (denominadas $P$).


El uso de esta notación obedece a una analogía estructural con los orbitales atómicos: el modo 
$S$ posee simetría axial y carece de nodos transversales, mientras que el modo $P$ presenta una estructura lobular con un nodo en su eje transversal. Esta nomenclatura se adopta como un recurso visual y conceptual, pero no implica la existencia de potenciales centrales ni de cuantización de momento angular como en el caso atómico. Su objetivo es establecer un lenguaje intuitivo que permita comunicar la estructura espacial y la simetría de los modos ópticos, y facilitar su análisis en configuraciones complejas.

Esta analogía se extiende al uso de términos como \textit{moléculas fotónicas}, con los que se describe un sistema de dos guías ópticas fuertemente acopladas, cuyas funciones propias se combinan para generar modos híbridos simétricos y antisimétricos. Esta idea toma inspiración de la química cuántica, donde los orbitales atómicos se combinan linealmente para formar orbitales moleculares enlazantes ($\sigma$) y antienlazantes ($\sigma^*$). En esta tesis, llamaremos \textit{molécula fotónica} a un par de guías acopladas cuya hibridación modal produce estados colectivos análogos a estos orbitales.

Es importante subrayar que esta es una analogía formal y geométrica. No se implica la existencia de electrones compartidos, fuerzas de enlace, niveles cuánticos ligados a configuraciones electrónicas, ni correlaciones cuánticas típicas del mundo molecular. No se observan efectos como energía de enlace, reglas de selección, ni cuantización de estados electrónicos. El término se utiliza como una metáfora visual y estructural que facilita la comprensión del acoplamiento modal y su rol como bloque constructivo en redes fotónicas más complejas. Este marco conceptual ha sido utilizado en diversos trabajos bajo el nombre de \textit{moléculas fotónicas} \citep{molecules}, y aquí se adopta con fines comunicativos, pero con una delimitación clara de su alcance físico.

En la misma línea, se utiliza la terminología de \textit{enlaces $\sigma$ y $\pi$} para referirse a los tipos de acoplamiento modal entre guías: el acoplamiento tipo $\sigma$ ocurre por superposición frontal (axial) de modos, mientras que el tipo $\pi$ surge por acoplamiento lateral, con nodos en el eje central. Esta caracterización se basa en la geometría de los campos ópticos involucrados, y será discutida en detalle en los capítulos correspondientes.

Uno de los fenómenos experimentales que motiva este estudio es el \textit{ángulo de invisibilidad} \cite{Pmodecoupling}, descrito en el Capítulo~\ref{cap:invisibility}. En ciertas configuraciones geométricas, los modos $P$ se desacoplan completamente unos de otros, permaneciendo ópticamente \textit{invisibles}. Este efecto revela que la simetría transversal de los modos puede ser explotada activamente como herramienta de control, permitiendo nuevas formas de localización y direccionamiento de la luz.

Por otro lado, también se analiza la ruptura de simetría en los acoplamientos evanescentes entre guías con diferente índice de refracción, lo que da lugar a una dinámica no recíproca. Aunque tradicionalmente se asume que el acoplamiento $C_{i\to j}$ entre dos guías es igual a $C_{j\to i}$, esta simetría puede romperse incluso en sistemas pasivos, generando acoplamientos asimétricos que se modelan mediante matrices no hermíticas con espectros reales. Este fenómeno se aborda en el Capítulo~\ref{cap:asymmetric}.

Finalmente, se introduce el concepto de acoplamiento interorbital $S\hspace{-0.3ex}P$, que permite hibridar modos con simetría distinta mediante la sintonización precisa de sus constantes de propagación. Este mecanismo se implementa experimentalmente en el Capítulo~\ref{cap:molecules}, donde se diseñan y caracterizan redes tipo SP-SSH: una extensión multiorbital del modelo de Su-Schrieffer-Heeger \cite{ssh}, con dos grados de libertad independientes que generan una doble transición topológica. El concepto de transición topológica se analiza desde la perspectiva de la teoría de bandas, la existencia de modos de borde y la polarización de bulto.

Este enfoque complementa y refuerza el eje general de esta tesis: estudiar cómo la estructura interna de los modos —ya sea su simetría transversal u orbital, así como la diferencia entre ellos— puede utilizarse como recurso físico para controlar la propagación de luz en redes fotónicas.

\vspace{1em}

Esta tesis se organiza de la siguiente manera:

El Capítulo~\ref{cap:teo} introduce el formalismo teórico utilizado a lo largo de la tesis, incluyendo los principios de propagación modal, la teoría de modos acoplados en redes discretas, y los conceptos topológicos relevantes en el estudio de bandas fotónicas.

En el Capítulo~\ref{cap:num} se presentan las herramientas numéricas empleadas. Se describe el método de Expansión en Modos Normales (EME), que permite modelar la evolución estacionaria de la luz, junto con otras metodologías complementarias como el método de propagación de haces (BPM) y la teoría de modos acoplados (CMT).

El Capítulo~\ref{cap:exp} detalla la técnica experimental de escritura láser por pulsos de femtosegundos, con énfasis en los parámetros clave para el diseño y fabricación de redes fotónicas multiorbitales tridimensionales.

En el Capítulo~\ref{cap:invisibility} se estudia el fenómeno de invisibilidad de los modos \( P \), caracterizando experimentalmente la geometría crítica que conduce al desacoplamiento óptico de estos modos. Esta propiedad se prueba en una red fotónica tipo grafeno. Para lograr una concordancia cuantitativa entre teoría y experimento, se introduce un modelo extendido que incorpora tanto la \textit{no ortogonalidad} modal como \textit{acoplamientos de largo alcance} entre guías no adyacentes.

El Capítulo~\ref{cap:asymmetric} examina el fenómeno de acoplamiento evanescente no simétrico entre guías de distinto índice de refracción. Se presenta una corrección a la teoría de modos acoplados convencional que permite modelar estos efectos mediante matrices de acoplamiento no hermíticas pero con espectros reales, y se valida experimentalmente el desbalance de intensidad resultante en dímeros asimétricos.

En el Capítulo~\ref{cap:molecules} se analiza la formación de modos híbridos mediante moléculas fotónicas para su uso como bloques constructivos en la formación de redes multiorbitales. Sobre esta base se implementa y caracteriza una red tipo SP-SSH, una extensión multiorbital del modelo de Su-Schrieffer-Heeger, incluyendo el estudio de sus fases topológicas mediante polarización de bulto y espectros de borde.

Finalmente, el Capítulo~\ref{cap:conclu} presenta las conclusiones del trabajo y discute posibles extensiones hacia otras plataformas fotónicas.
