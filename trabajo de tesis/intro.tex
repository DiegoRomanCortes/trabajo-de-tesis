\chapter{Introducción}

Durante la última década, varios Premios Nobel en Física han estado estrechamente ligados a la óptica \cite{nobel}: por la generación de pulsos ultracortos de luz —femtosegundos \cite{femto1} y luego attosegundos \cite{atto1, atto2, atto3}—, por experimentos pioneros con fotones entrelazados \cite{photons1, photons2, photons3}, por el desarrollo de pinzas ópticas \cite{opticaltweezers} y por la invención de LEDs de alta eficiencia \cite{led1, led2, led3}. Estos avances fundamentales han impulsado aplicaciones industriales, biomédicas, en telecomunicaciones e incluso en defensa. Una aplicación cotidiana destacada es la fibra óptica, que actúa como guía de onda para la luz y constituye hoy el principal medio de transmisión de datos a nivel global \cite{fibra2, fibra}.

Muchos de estos desarrollos se han visto potenciados por la técnica de \textit{escritura láser de femtosegundos}, que permite fabricar redes fotónicas tridimensionales de guías de onda con precisión micrométrica. Esta tecnología ha habilitado la realización experimental de redes con geometrías complejas y grados de libertad artificiales, y ha sido utilizada para estudiar fenómenos como oscilaciones de Bloch \cite{BlochOsci}, localización de Anderson \cite{Anderson}, dinámica de bandas planas \cite{lieb1, lieb2, artificialFB, FBdynamics}, fases topológicas \cite{obstopo, obsfloquet, topo1dphoto, toporusos}, formación de solitones no lineales \cite{discretesolitons} e incluso la propagación de luz cuántica \cite{qed, squeezed, topoquantum}.

Dentro de este marco, el presente trabajo se enfoca en el estudio de redes fotónicas \textit{multiorbitales}, en las que los modos transversales de la luz —en particular los modos \( P \), antisimétricos— actúan como grados de libertad internos que pueden aprovecharse para enriquecer la dinámica óptica. A diferencia de los sistemas convencionales, donde se propaga solamente el modo fundamental (\( S \)), las redes multiorbitales permiten emular estructuras más complejas, como acoplamientos interorbitales o interferencias moduladas por la geometría.

Uno de los fenómenos experimentales que motiva este estudio es el \textit{ángulo de invisibilidad} \cite{Pmodecoupling}, descrito en el Capítulo~\ref{cap:invisibility}. En ciertas condiciones geométricas, los modos \( P \) se desacoplan completamente unos de otros, permaneciendo ópticamente \textit{invisibles}. Este fenómeno pone de manifiesto que la simetría transversal de los modos no solo es relevante, sino que puede ser usada activamente como herramienta de control. Este principio guía la exploración de nuevas formas de localización, filtrado y direccionamiento de luz en dispositivos fotónicos integrados.

Además, se aborda en esta tesis un fenómeno más sutil pero igualmente relevante: la \textit{ruptura de simetría en los acoplamientos evanescentes} entre guías. En sistemas reales, el supuesto estándar de que el acoplamiento entre dos modos es simétrico —es decir, que \( C_{i \to j} = C_{j \to i} \)— puede dejar de ser válido cuando las guías presentan diferentes perfiles de índice de refracción. Este efecto, que no requiere pérdida, ganancia ni elementos activos, da lugar a una dinámica no recíproca observable experimentalmente. Dicho marco teórico se presenta en el Capítulo~\ref{cap:asymmetric}, junto con validación experimental basada en redes tipo dímero con distintos contrastes de índice.

Luego, se utiliza el concepto de \textit{acoplamiento interorbital SP}, un mecanismo que permite hibridar modos \( S \) y \( P \) ajustando finamente sus constantes de propagación a través del control de la potencia del láser de escritura \cite{interorbital}. Este acoplamiento da lugar a una dinámica rica, especialmente cuando se implementa en redes periódicas. En el Capítulo~\ref{cap:molecules}, este principio se aplica a la construcción de moléculas fotónicas —pares de guías fuertemente acopladas— cuyas combinaciones simétrica y antisimétrica funcionan como orbitales efectivos. Sobre esta base, se implementa experimentalmente el modelo \textit{SP-SSH} \cite{SPSSH}, una extensión multiorbital del modelo de Su-Schrieffer-Heeger \citep{ssh}. Este sistema presenta una \textit{doble transición de fase topológica}, determinada por dos parámetros independientes: la dimerización geométrica y el desintonizado de las constantes de propagación entre los $S$ y $P$. La existencia de tres fases distintas —con dos, cuatro o ningún modo de borde— se analiza mediante la polarización de bulto y técnicas espectrales.

 

Esta línea complementa y refuerza el eje general de esta tesis: estudiar cómo la estructura interna de los modos —ya sea su simetría transversal, orbital o evanescente— puede utilizarse como recurso físico para controlar la propagación de luz en redes fotónicas.


\vspace{1em}

Este trabajo se organiza de la siguiente manera:

El Capítulo~\ref{cap:teo} introduce el formalismo teórico utilizado a lo largo de la tesis, incluyendo los principios de propagación modal, la teoría de modos acoplados en redes discretas, y los conceptos topológicos relevantes para bandas ópticas.

En el Capítulo~\ref{cap:num} se presentan las herramientas numéricas empleadas. Se describe el método de Expansión en Modos Normales (EME), que permite modelar la evolución estacionaria de la luz, junto con otras metodologías complementarias como el método de propagación de haces (BPM) y la teoría de modos acoplados (CMT).

El Capítulo~\ref{cap:exp} detalla la técnica experimental de escritura láser por pulsos de femtosegundos, con énfasis en los parámetros clave para el diseño y fabricación de redes fotónicas multiorbitales tridimensionales.

En el Capítulo~\ref{cap:invisibility} se estudia el fenómeno de invisibilidad modal de los modos \( P \), caracterizando experimentalmente la geometría crítica que conduce al desacoplamiento óptico de estos modos. Esta propiedad se prueba en una red tipo panal de abeja. Para lograr una concordancia cuantitativa entre teoría y experimento, se introduce un modelo extendido que incorpora tanto la \textit{no ortogonalidad} modal como \textit{acoplamientos de largo alcance} entre guías no adyacentes.

El Capítulo~\ref{cap:asymmetric} examina el fenómeno de acoplamiento evanescente no simétrico entre guías de distinto índice de refracción. Se presenta una corrección a la teoría de modos acoplados convencional que permite modelar estos efectos mediante matrices de acoplamiento no hermíticas pero con espectros reales, y se valida experimentalmente el desbalance de intensidad resultante en dímeros asimétricos.

En el Capítulo~\ref{cap:molecules} se analiza la formación de moléculas fotónicas como bloques constructivos para redes multiorbitales. Sobre esta base se implementa y caracteriza una red tipo SP-SSH, una extensión multiorbital del modelo de Su-Schrieffer-Heeger, incluyendo el estudio de sus fases topológicas mediante polarización de bulto y espectros de borde.

Finalmente, el Capítulo~\ref{cap:conclu} presenta las conclusiones del trabajo y discute posibles extensiones hacia otras plataformas fotónicas.
