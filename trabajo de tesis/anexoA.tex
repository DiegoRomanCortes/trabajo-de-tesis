\chapter{Ortogonalidad de los modos normales \label{sec:orto}}

La ortogonalidad de los modos normales $\textbf{E}_\nu^\perp$ se puede demostrar usando el convenio de Einstein en la ecuación (\ref{eqn:eigenfield}) para simplificar la notación:
\begin{equation}
	T_{ij} E^\nu_j = \beta_\nu^2 E^\nu_i, \label{eqn:eigentensorial}
\end{equation}
donde $T_{ij}E_j \equiv \delta_{ij}\left(\nabla_\perp^2 + k_0^2n^2\right)E_j+\partial_i \left(E_j  \partial_j\ln(n^2) \right) $. Es un poco más sencillo demostrar que $T_{ij}$ es un operador hermítico, es decir, que:
\begin{align*}
	\iint \left(T_{ij} E_j^\mu \right)^*  E_i^\nu dA = \iint \left( E_i^\mu \right)^*  \left(T_{ij} E_j^\nu\right) dA .
\end{align*}
Equivalentemente, notemos que
\begin{align*}
	\left(T_{ji} E_i^\mu \right)^* E_j^\nu  - \left( E_i^\mu \right)^*  \left(T_{ij} E_j^\nu \right) &= \nabla_\perp \cdot \left[E_i^\nu \nabla_\perp \left( E_i^\mu \right)^* - \left( E_i^\mu \right)^* \nabla_\perp E_i^\nu \right] 
	\\	
	&+ \left[E^\nu_j \partial_j \left( E_i^\mu \right)^*  - \left( E_j^\mu \right)^* \partial_j  E_i^\nu \right] \partial_i \ln(n^2).  
\end{align*}
Al integrar en el plano, la divergencia se convierte en un término de frontera que se anula al considerar modos guiados (que decaen a cero en el infinito), mientras que el segundo término es cero sólo si $\nabla n^2 = \textbf{0}$. La hermiticidad de $T_{ij}$ permite escoger una base ortogonal de modos normales para el campo eléctrico.