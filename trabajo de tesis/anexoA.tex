\section{Ortogonalidad de los modos normales \label{sec:orto}}

La ortogonalidad de los modos normales $\textbf{E}_\nu^\perp$ puede demostrarse utilizando el convenio de Einstein en la ecuación (\ref{eqn:eigenfield}) para simplificar la notación:
\begin{equation}
	T_{ij} E^\nu_j = \beta_\nu^2 E^\nu_i, \label{eqn:eigentensorial}
\end{equation}
donde el operador $T_{ij}$ actúa como $T_{ij}E_j \equiv \delta_{ij}\left(\nabla_\perp^2 + k_0^2n^2\right)E_j + \partial_i \left(E_j \partial_j\ln(n^2)\right).$

Para demostrar la ortogonalidad de los modos $E_i$, se puede demostrar equivalentemente que $T_{ij}$ es hermítico:
\begin{align}
	\iint \left(T_{ij} E_j^\mu\right)^* E_i^\nu \,dA = \iint \left(E_i^\mu\right)^* \left(T_{ij} E_j^\nu\right) \,dA.
\end{align}
Esta propiedad se sigue de la identidad:
\begin{align*}
	&\left(T_{ji} E_i^\mu\right)^* E_j^\nu - \left(E_i^\mu\right)^* \left(T_{ij} E_j^\nu\right) =  \nabla_\perp \cdot \left[E_i^\nu \nabla_\perp \left(E_i^\mu\right)^* - \left(E_i^\mu\right)^* \nabla_\perp E_i^\nu\right]  + \left[E^\nu_j \partial_j \left(E_i^\mu\right)^* - \left(E_j^\mu\right)^* \partial_j E_i^\nu\right] \partial_i \ln(n^2).
\end{align*}
Al integrar sobre el plano transversal el término de divergencia se anula para modos guiados (que decaen en el infinito), pero el segundo término desaparece sólo si $\nabla n^2 = \textbf{0}$. Es decir, el operador $T_{ij}$ sólo es Hermítico en la condición de guiaje débil.

El sistema completo es hermítico, como se demuestra partiendo de la ecuación (\ref{eqn:rotordoble}):
\begin{align*}
	\iiint_V \left(\nabla\times\nabla\times\textbf{E}^\nu\right) \cdot \left(\textbf{E}^\mu\right)^* dV &= \iiint_V n^2\frac{\omega_\nu^2}{c^2} \textbf{E}^\nu \cdot \left(\textbf{E}^\mu\right)^* dV \\
	\iiint_V \left[\nabla\times\nabla\times\left(\textbf{E}^\mu\right)^*\right] \cdot \textbf{E}^\nu dV &= \iiint_V n^2 \frac{\omega_\mu^2}{c^2} \left(\textbf{E}^\mu\right)^* \cdot \textbf{E}^\nu dV
\end{align*}

Restando ambas ecuaciones y aplicando el teorema de divergencia:
\begin{align*}
	\frac{\omega_\nu^2 - \omega_\mu^2}{c^2} \iiint_V n^2 \textbf{E}^\nu \cdot \left(\textbf{E}^\mu\right)^* dV &= \oiint_{\partial V} \left[ \cdots \right] \cdot \hat{\textbf{n}}\,dA \\
	&\xrightarrow{\partial V \to \infty} 0
\end{align*}
Se obtienen así dos posibilidades: o los modos son degenerados ($\omega_\nu^2 = \omega_\mu^2$), o se tiene que $\iiint n^2 \textbf{E}^\nu \cdot \left(\textbf{E}^\mu\right)^* dV = 0$.