\chapter{Ortogonalidad de los modos normales \label{sec:orto}}

La ortogonalidad de los modos normales $\textbf{E}_\nu^\perp$ se puede demostrar usando el convenio de Einstein en la ecuación (\ref{eqn:eigenfield}) para simplificar la notación:

\begin{equation}
	\left(\nabla_\perp^2 + k_0^2n^2\right) E^\nu_i + G_{ij} E^\nu_j = \beta_\nu^2 E^\nu_i,
\end{equation}
donde $G_{ij}E_j \equiv \partial_i \left(E_j  \partial_j\ln(n^2) \right)$. Al tomar producto punto con $\left(\textbf{E}_\mu^\perp\right)^*$, intercambiar los índices, tomar conjugado y restar, se tiene:
\begin{align*}
	(\beta_\nu^2-\beta_\mu^2)E^\nu_i \left(E^\mu\right)^*_i &=  \left(E^\mu\right)^*_i \nabla_\perp^2 E^\nu_i - E^\mu_i \nabla_\perp^2 \left(E^\mu\right)^*_i  + \left(E^\mu\right)^*_i G_{ij} E^\nu_j - E^\mu_i G_{ij} \left(E^\mu\right)^*_j
	\\
	&= \nabla \cdot \left[ \left(E^\mu\right)^*_i \nabla_\perp E^\nu_i - E^\nu_i \nabla_\perp \left(E^\mu\right)^*_i \right] + \partial_i \left\{\left[ \left(E^\mu\right)^*_i  E^\nu_j  - E^\nu_i \left(E^\mu\right)^*_j \right] \partial_j   \ln(n^2) \right\}
	\\
	&-\left[ E^\nu_j \partial_i \left(E^\mu\right)^*_i -   \left(E^\mu\right)^*_j \partial_i E^\nu_i\right] \partial_j\ln(n^2)
\end{align*}
Al integrar en el plano transversal, rigr el teorema de la divergencia. Como los campos deben decaer a cero en el infinito, el lado derecho se simplifica a:
\begin{equation}
	(\beta_\nu^2-\beta_\mu^2)\iint \textbf{E}^\perp_\nu \cdot \left(\textbf{E}^\perp_\mu\right)^* dx dy = \iint  \left[ \left(E^\mu\right)^*_j \partial_i E^\nu_i - E^\nu_j \partial_i \left(E^\mu\right)^*_i\right] \partial_j\ln(n^2) dxdy \approx
	0.
\end{equation}
Es decir, si $\beta_\nu \neq \beta_\mu$, los modos son ortogonales en guiaje débil.