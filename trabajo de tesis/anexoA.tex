\chapter{Ortogonalidad de los modos normales \label{sec:orto}}

La ecuación (\ref{eqn:eigenfield}) permite en encontrar la condición de ortogonalidad de los modos normales del campo eléctrico  $\textbf{E}_\nu$. Para ello, se puede tomar producto punto a la ecuación por $\textbf{E}_\mu^*$ e integrar en una superficie $S$:

\begin{align*}
	\begin{pmatrix}
	\nabla_\perp^2  + k_0^2n^2 + G_{xx} & G_{xy} 
	\\
	G_{yx}  & \nabla_\perp^2  + k_0^2n^2 + G_{yy} 
	\end{pmatrix}
\textbf{E}^\perp_\nu
		&=
	\beta_\nu^2 \textbf{E}^\perp_\nu
\end{align*}

\begin{align*}
	 \iint_S \begin{pmatrix}
	\nabla_\perp^2  + k_0^2n^2 + G_{xx} & G_{xy} 
	\\
	G_{yx}  & \nabla_\perp^2  + k_0^2n^2 + G_{yy} 
	\end{pmatrix}
\textbf{E}^\perp_\nu \cdot\textbf{E}_\mu^* dxdy &= \beta_\nu^2 \iint_S \textbf{E}_\nu \cdot\textbf{E}_\mu^* dxdy
\end{align*}

\begin{align}
	 \iint_S \left(  \nabla_\perp^2  + k_0^2n^2 \right)\textbf{E}_\nu \cdot\textbf{E}_\mu^* dxdy &= \beta_\nu^2 \iint_S \textbf{E}_\nu \cdot\textbf{E}_\mu^* dxdy \label{eqn:munu}
\end{align}
Si intercambiamos los índices $\nu$ y $\mu$ tenemos:
\begin{align}
	 	\iint_S \left(  \nabla_\perp^2  + k_0^2n^2 \right)\textbf{E}_\mu \cdot\textbf{E}_\nu^* dxdy &= \beta_\mu^2 \iint_S \textbf{E}_\mu \cdot\textbf{E}_\nu^* dxdy \label{eqn:numu}
\end{align}
Restando las ecuaciones (\ref{eqn:munu}) y (\ref{eqn:numu}) y escogiendo modos reales:
\begin{align*}
	 	  (\beta_\nu^2-\beta_\mu^2)\iint_S\textbf{E}_\mu \cdot\textbf{E}_\nu dxdy &=  \iint_S \left(\textbf{E}_\mu   \cdot\nabla_\perp^2 \textbf{E}_\nu - \textbf{E}_\nu   \cdot\nabla_\perp^2 \textbf{E}_\mu\right)  dxdy
		\nonumber	 	  
	 	  \\
	 	  &=
	 	  \oint_{\partial S} (\textbf{E}_\mu \times \nabla \times \textbf{E}_\nu - \textbf{E}_\nu \times \nabla \times \textbf{E}_\mu) \cdot \hat{n}d\ell
	 	  \nonumber
			\\	 	  
	 	  &+
	 	  \oint_{\partial S} (\hat{n}\cdot \textbf{E}_\mu)(\nabla\cdot \textbf{E}_\nu) - (\hat{n}\cdot \textbf{E}_\nu)(\nabla\cdot \textbf{E}_\mu) d\ell
	 	  \nonumber
	 	  \\
	 	  &\underset{\mathrm{\partial S \to \infty}}{=}  0.
	 	 \\
	 	 \therefore \iint_S\textbf{E}_\mu \cdot\textbf{E}_\nu^* dxdy &= 0, \quad \text{si } \nu\neq\mu,
\end{align*}
donde se ha usado la segunda identidad de Green vectorial \citep{greenvectorial} y que el campo eléctrico debe hacerse anularse en el infinito, aunque basta considerar que el campo sea nulo en el borde de la región $S$.