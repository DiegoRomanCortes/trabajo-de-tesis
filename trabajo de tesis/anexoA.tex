\chapter{Ortogonalidad de los modos normales \label{sec:orto}}

La ortogonalidad de los modos normales $\textbf{E}_\nu^\perp$ se puede demostrar usando el convenio de Einstein en la ecuación (\ref{eqn:eigenfield}) para simplificar la notación:
\begin{equation}
	\left(\nabla_\perp^2 + k_0^2n^2\right) E^\nu_i + G_{ij} E^\nu_j = \beta_\nu^2 E^\nu_i, \label{eqn:eigentensorial}
\end{equation}
donde $G_{ij}E_j \equiv \partial_i \left(E_j  \partial_j\ln(n^2) \right) = -\partial_i \partial_j E_j  $. Al tomar producto punto con $\left(\textbf{E}_\mu^\perp\right)^*$, intercambiar los índices, tomar conjugado y restar la ecuación (\ref{eqn:eigentensorial}), se tiene:
\begin{align*}
	(\beta_\nu^2-\beta_\mu^2)E^\nu_i \left(E^\mu\right)^*_i &=  \left(E^\mu\right)^*_i \nabla_\perp^2 E^\nu_i - E^\nu_i \nabla_\perp^2 \left(E^\mu\right)^*_i  + \left(E^\mu\right)^*_i G_{ij} E^\nu_j - E^\nu_i G_{ij} \left(E^\mu\right)^*_j
	\\
	&= \nabla \cdot \left[ \left(E^\mu\right)^*_i \nabla_\perp E^\nu_i - E^\nu_i \nabla_\perp \left(E^\mu\right)^*_i \right] + \partial_i \left\{ E_i^\nu \partial_j \left(E^\mu\right)^*_j - \left(E^\mu\right)^*_i \partial_j E_j^\nu  \right\}
\end{align*}
Al integrar en el plano transversal, se puede integrar por partes y notar que los términos son de frontera. Considerando que los campos deben anularse en el infinito, se tiene que:
\begin{equation}
	(\beta_\nu^2-\beta_\mu^2)\iint \textbf{E}^\perp_\nu \cdot \left(\textbf{E}^\perp_\mu\right)^* dx dy = 0.
\end{equation}
Es decir, si $\beta_\nu^2 \neq \beta_\mu^2$, los modos son ortogonales.