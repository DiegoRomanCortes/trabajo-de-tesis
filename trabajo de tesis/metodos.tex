\chapter{Métodos experimentales}

\section{Escritura de guías de onda}
\begin{figure}[H]
	\centering
	\includegraphics[width=0.6\linewidth, trim={18cm 4cm 15cm 6cm},clip]{media/fabrication}
	\caption{Escritura.}
\end{figure}

\section{Montaje de excitación láser supercontinuo}
\begin{figure}[H]
	\centering
	\includegraphics[width=\linewidth, trim={5cm 9cm 3cm 7cm},clip]{media/SC_setup}
	\caption{Montaje SC.}
\end{figure}
\newpage
\section{Montaje de modulación espacial de luz}

Para usar condiciones iniciales distintas a una gaussiana se hace necesario incorporar métodos de modulación espacial de luz. En esta tesis se utilizó una técnica conocida como 

\subsection{Etapa premodulación}
	El modulador espacial de luz utilizado es un HOLOEYE PLUTO-NIR SLM -  Reflective LCOS, cuya respuesta óptica ocurre con polarización paralela al plano de la mesa óptica. Se utiliza un retardador de media onda ($\lambda/2$) para que la polarización del la luz láser coincida con la de la respuesta del SLM. Posteriormente se magnifica y se colima el haz para que abarque todo el área de pixeles disponible.
\subsection{Etapa de modulación}
\subsection{Etapa de acoplamiento}
\subsection{Etapa de captura en cámara}

\subsection{Circuito óptico}
\begin{figure}[H]
	\centering
	\includegraphics[width=\linewidth, trim={21cm 5cm 7cm 5cm},clip]{media/SLM_setup}
	\caption{Montaje SLM.}
\end{figure}
\section{Análisis de imágenes}

