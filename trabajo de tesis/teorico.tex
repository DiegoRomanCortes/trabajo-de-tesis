\chapter{Marco teórico}

\section{Desde las ecuaciones de Maxwell a propagación de la luz en guías de onda dieléctricas}

Esta tesis estudia el comportamiento de luz láser de baja potencia (1 mW de potencia de salida) propagada en guías de onda dieléctricas escritas dentro de una muestra de borosilicato. Es por ello que se supone un medio lineal no magnético libre de fuentes de carga y de corriente. Las ecuaciones de Maxwell (mks) en este régimen son:
\begin{align}
	\nabla\cdot\textbf{D} &= 0, \label{eqn:gauss}
	\\	
	\nabla\times\textbf{E} &= -\frac{\partial \textbf{B}}{\partial t}, \label{eqn:faraday-lenz}
	\\	
	\nabla\cdot\textbf{B} &= 0,
	\\	
	\nabla\times\textbf{B} &= \mu_0\frac{\partial \textbf{D}}{\partial t}, \label{eqn:ampere-maxwell}
\end{align}
donde \textbf{E}, \textbf{B}, $\textbf{D}=\varepsilon\textbf{E}$ son los campos eléctrico, campo magnético y campo desplazamiento eléctrico, respectivamente. Las guías de onda son invariantes en la dirección de propagación $z$, por lo que el índice de refracción $n=\sqrt{\varepsilon/\varepsilon_0}$ dependerá de las coordenadas transversales al eje óptico, es decir, $n \equiv n(x,y) = n_0 + \Delta n(x,y)$, con $n_0=1.47$ el índice de refracción del borosilicato y $\Delta n \sim 10^{-5}-10^{-3}$ el contraste de las guías de onda.

Aplicando rotor por la izquierda a la ecuación de Faraday-Lenz (\ref{eqn:faraday-lenz}), usando la ecuación de Ampère-Maxwell (\ref{eqn:ampere-maxwell}) y asumiendo una solución temporal harmónica proporcional a $e^{i\omega t}$ se tiene:

\begin{align}
	\nabla\times\nabla\times\textbf{E} &= -\frac{\partial}{\partial t}(\nabla\times\textbf{B}) = -\frac{\partial}{\partial t}\left(\mu_0\frac{\partial \textbf{D}}{\partial t}\right) = -\frac{n^2}{c^2}\frac{\partial^2 \textbf{E}}{\partial t^2} = n^2k_0^2 \textbf{E}, \label{eqn:rotordoble}
\end{align}
donde $k_0 \equiv \omega/c$ es el número de onda en el vacío. Notemos que, por identidad de cálculo vectorial, se tiene que $\nabla\times\nabla\times\textbf{E} = \nabla(\nabla\cdot\textbf{E}) - \nabla^2\textbf{E}$, y usando la ley de Gauss (\ref{eqn:gauss}) se deduce que $\nabla\cdot \textbf{E} = -\nabla(n^2)\cdot\textbf{E}/n^2$.

Con esto, 

\begin{equation}
	(\nabla^2  + k_0^2n^2)\textbf{E} = -\nabla\left(\textbf{E} \cdot \frac{\nabla n^2}{n^2}\right) \approx 0, \label{eqn:helmholz}
\end{equation}

donde el término de la derecha se desprecia usando la aproximación de guiaje débil, pues el contraste $\Delta n$ es pequeño. 

\section{Soluciones analíticas para guía de onda tipo losa o \textit{slab}}




\section{Modos normales en guías de onda}

Si la estructura de guías de onda no varía en la dirección $z$, el campo eléctrico se puede expresar como una onda plana del tipo $\textbf{E}(\textbf{r}) = \textbf{E}_\nu(x, y) e^{i\beta_\nu z}$. A su vez, el laplaciano de la ecuación (\ref{eqn:helmholz}) se puede separar como $\nabla^2 \equiv \nabla_\perp^2 + \frac{\partial^2}{\partial z^2}$, por lo que la ecuación (\ref{eqn:helmholz}) se expande como:

\begin{align}
	(\nabla^2  + k_0^2n^2) \textbf{E}(\textbf{r}) &= \left(\nabla_\perp^2 + \frac{\partial^2}{\partial z^2} + k_0^2n^2\right) \textbf{E}_\nu(x, y)  e^{i\beta_\nu z} \nonumber
\\	
	&= e^{i\beta_\nu z} \nabla_\perp^2 \textbf{E}_\nu -\beta_\nu^2\textbf{E}_\nu e^{i\beta_\nu z} + k_0^2n^2 \textbf{E}_\nu  e^{i\beta_\nu z}
\nonumber	
	\\	
	&= \left[  \nabla_\perp^2  + (k_0^2n^2-\beta_\nu^2) \right]\textbf{E}_\nu  e^{i\beta_\nu z}
	\nonumber	
	\\
	&=
	0
	\nonumber
	\\
	\therefore
	 \left[  \nabla_\perp^2  + k_0^2n^2(x,y) \right]&\textbf{E}_\nu(x,y)  = \beta_\nu^2 \textbf{E}_\nu(x,y) \label{eqn:eigenfield}
\end{align}
Notemos que la ecuación (\ref{eqn:eigenfield}) es un problema de autovalores $\beta_\nu^2$ y autofunciones $\textbf{E}_\nu(x,y)$, que son ortogonales y forman una base completa (ver apéndice \ref{sec:orto}). En principio, la forma espacial del índice de refracción $n(x, y)$ puede ser arbitraria siempre y cuando que se satisfaga la condición de guiaje débil de la ecuación (\ref{eqn:helmholz}). 

\section{Teoría de modos acoplados}
	Para el estudio de redes fotónicas, es conveniente utilizar herramientas similares a las de la Física del Sólido en lo que respecta a potenciales periódicos. En particular, se puede suponer que los modos guiados de una guía de onda están fuertemente ligados a ella (enlace fuerte o \textit{Tight Binding}), incluso en presencia de otras guías de onda. Es decir, se supondrá que el $\nu$-ésimo modo de la $m$-ésima guía de onda satisface para toda distancia de propagación $z$ la ecuación (\ref{eqn:eigenfield}), donde el índice de refracción total se puede descomponer en una suma periódica de guías de onda $n^2(\textbf{r}) = \sum_{m} n^2_m(\textbf{r})$. Entonces, descomponiendo el campo eléctrico total de la forma $\textbf{E}(\textbf{r}) = \sum_{\nu, m} \textbf{E}_{\nu, m}(x, y) a_{\nu, m}(z) e^{i\beta_{\nu, m} z}$ y reemplazando en la ecuación (\ref{eqn:helmholz}) se tiene:

\begin{align}
	(\nabla^2  + k_0^2n^2) \textbf{E}(\textbf{r}) &= \left(\nabla_\perp^2 + \frac{\partial^2}{\partial z^2} + k_0^2n^2 \right)\sum_{\nu, m} \textbf{E}_{\nu, m}(x, y) a_{\nu, m}(z) e^{i\beta_{\nu, m} z}
	\nonumber
	\\
	&= \sum_{\nu, m} \left[a_{\nu, m} e^{i\beta_{\nu, m} z} \left(\nabla_\perp^2 +k_0^2n^2 \right)\textbf{E}_{\nu, m} + \textbf{E}_{\nu, m}\frac{d^2}{d z^2}a_{\nu, m} e^{i\beta_{\nu, m} z}\right]
	\nonumber	
	\\
	&= \sum_{\nu, m} \left[a_{\nu, m}  \left(\nabla_\perp^2 +k_0^2n^2 -\beta_{\nu,m}^2 \right) + \frac{d^2 a_{\nu, m}}{d z^2}  +2i\beta_{\nu,m}\frac{d a_{\nu, m}}{d z} \right]e^{i\beta_{\nu, m} z}\textbf{E}_{\nu, m}
		\nonumber	
	\\
	&\approx \sum_{\nu, m} \left[a_{\nu, m}  k_0^2(n^2 - n^2_{m}) +2i\beta_{\nu,m}\frac{d a_{\nu, m}}{d z} \right]e^{i\beta_{\nu, m} z}\textbf{E}_{\nu, m}
	\nonumber	
	\\
	&= 0.
	\nonumber	
\end{align}
Aplicando producto punto con $\textbf{E}_{\mu, m'}^*$ e integrando en todo el plano $xy$:
\begin{align}
	  \int\displaylimits_{-\infty}^{+\infty}\int\displaylimits_{-\infty}^{+\infty} \sum_{\nu, m} \left[a_{\nu, m}  k_0^2(n^2 - n^2_{m}) +2i\beta_{\nu,m}\frac{d a_{\nu, m}}{d z} \right]e^{i\beta_{\nu, m} z}\textbf{E}_{\nu, m} \cdot \textbf{E}_{\mu, m'}^* dxdy &= 0
	  \nonumber
	  \\
	  \sum_{\nu, m} \left[2i\beta_{\nu,m}\frac{d a_{\nu, m}}{d z} \delta_{\nu,\mu}\delta_{m,m'} +  2\beta_{\mu, m'}C_{m, m', \nu, \mu}   a_{\nu, m} \right]e^{i\beta_{\nu, m} z} &= 0
	  \nonumber
	  \\
	  	  i\frac{d a_{\mu, m'}}{d z} e^{i\beta_{\mu, m'} z} +  \sum_{\nu, m\neq m'}C_{m, m', \nu, \mu}   a_{\nu, m} e^{i\beta_{\nu, m} z} &= 0
	\label{eqn:CMT1}
\end{align}
donde se han definido y usado
\begin{align*}
	   \int\displaylimits_{-\infty}^{+\infty}\int\displaylimits_{-\infty}^{+\infty} k_0^2(n^2 - n^2_{m}) \textbf{E}_{\nu, m} \cdot \textbf{E}_{\mu, m'}^* dxdy & \approx 2\beta_{\mu, m'}C_{m, m', \nu, \mu}  
	\nonumber
	\\
	\int\displaylimits_{-\infty}^{+\infty}\int\displaylimits_{-\infty}^{+\infty} \textbf{E}_{\nu, m} \cdot \textbf{E}_{\mu, m'}^* dxdy &\approx \delta_{\nu,\mu}\delta_{m,m'},
\end{align*}
es decir, que el efecto del modo $(\nu, m)$ en la dinámica del modo $(\mu, m')$ sólo es apreciable al ponderar con la expresión $(n^2 - n^2_{m})$, lo que da origen al término $C_{m, m', \nu, \mu}$ conocido comúnmente como acoplamiento. Sin el peso del contraste, la interacción es evanescente, por lo que la aproximación de ortogonalidad se hace razonable con suficiente distancia entre guías (sobre los 15 $\mu$m en los experimentos de esta tesis). Cuando $m=m'$, el acoplamiento $C_{m, m', \nu, \mu}$ es nulo por definición. Es por ello que no se incluye en la sumatoria de la ecuación (\ref{eqn:CMT1}).

	
